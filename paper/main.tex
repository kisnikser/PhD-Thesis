\documentclass{dissert}
\usepackage[T2A]{fontenc}
\usepackage[utf8]{inputenc}
\usepackage[russian]{babel}

\usepackage{fullpage}
\usepackage{lastpage}
\usepackage{enumerate}
\usepackage{svg}

\usepackage[all]{xy}
% colors
\usepackage{xcolor}
\definecolor{darkgreen}{rgb}{0.0, 0.2, 0.13}
\definecolor{darkcyan}{rgb}{0.0, 0.55, 0.55}

\usepackage{geometry}
\geometry{left=2.5cm}
\geometry{right=1.0cm}
\geometry{top=2.0cm}
\geometry{bottom=2.0cm}
\renewcommand{\baselinestretch}{1.2}

\newcommand{\paragraph}[1]{\noindent\textbf{#1}\quad}

%https://tex.stackexchange.com/questions/163451/total-number-of-citations
\usepackage{totcount}
\newtotcounter{citnum} %From the package documentation
\def\oldbibitem{} \let\oldbibitem=\bibitem
\def\bibitem{\stepcounter{citnum}\oldbibitem}

\makeatletter
\long\def\@makecaption#1#2{%
  \vskip\abovecaptionskip
  \sbox\@tempboxa{#1.~#2}%
  \ifdim \wd\@tempboxa >\hsize
    #1.~#2\par
  \else
    \global \@minipagefalse
    \hb@xt@\hsize{\hfil\box\@tempboxa\hfil}%
  \fi
  \vskip\belowcaptionskip}
\makeatother

\reversemarginpar

\renewcommand{\contentsname}{Содержание}
\renewcommand{\contentsdesc}{Стр.}
\renewcommand{\chaptername}{Глава}

%%% Библиография %%%
\makeatletter
\bibliographystyle{utf8gost71u}     % Оформляем библиографию по ГОСТ 7.1 (ГОСТ Р 7.0.11-2011, 5.6.7)
\makeatother

\usepackage{silence}
\WarningFilter*{latex}{Text page \thepage\space contains only floats}

% Нужные мне пакеты
\usepackage[hidelinks]{hyperref}
\usepackage{booktabs}
\usepackage{diagbox}
\usepackage{amsthm}
\usepackage{mathrsfs}
\usepackage{graphicx}
\usepackage{placeins}
\usepackage{amsmath,amssymb,amsfonts}
\usepackage{graphicx}
\usepackage{caption}
\usepackage{subcaption}
\usepackage{color}
\usepackage{bm}
\usepackage{tabularx}
\usepackage{url}
\usepackage{multirow}
\usepackage{indentfirst}

\newtheorem{theorem}{Теорема}
\newtheorem{lemma}[theorem]{Лемма}
\newtheorem{definition}{Определение}
\newtheorem{assumption}{Предположение}
\newtheorem{property}{Свойство}
\newtheorem{corollary}{Следствие}
\newtheorem{remark}{Замечание}
\newtheorem{hypothesis}{Гипотеза}

% \usepackage{algorithm}
\usepackage{algpseudocode}
\usepackage[ruled,vlined]{algorithm2e}
\SetKwInOut{KwIn}{Input}
\SetKwInOut{KwOut}{Output}
\SetKwFor{ForEach}{for each}{do}{end}


\usepackage{comment}
\usepackage{rotating}

\usepackage{autonum}

\begin{document}

% Титульный лист
\thispagestyle{empty}

\begin{titlepage}

\begin{center}
Министерство науки и высшего образования Российской Федерации\\
Федеральное государственное автономное образовательное учреждение\\ высшего образования\\
<<Московский физико-технический институт\\
(национальный исследовательский университет)>>\\
Физтех-школа прикладной математики и информатики\\
Кафедра интеллектуальных систем
\end{center}

\vspace{0pt plus2fill}
\flushright{На правах рукописи}
\vspace{0pt plus2fill}

\begin{center}
Киселев Никита Сергеевич
\end{center}
\vspace{1cm}

\begin{center}
\textbf{О~СВЯЗИ ОПТИМИЗАЦИОННОЙ ПОВЕРХНОСТИ С~ДОСТАТОЧНЫМ РАЗМЕРОМ ВЫБОРКИ В~ПАРАМЕТРИЧЕСКИХ МОДЕЛЯХ МАШИННОГО ОБУЧЕНИЯ}
\end{center}
\vspace{1cm}

\begin{center}
1.2.1~--- Искусственный интеллект и машинное обучение
\end{center}
\vspace{1cm}

\begin{center}
Диссертация на соискание ученой степени\\кандидата физико-математических наук
\end{center}
\vspace{0pt plus2fill}

\begin{flushleft}
\hfill\begin{tabular}{l}
\textbf{Научный руководитель:}\\
Грабовой Андрей Валериевич,\\
канд. физ.-мат. наук
\end{tabular}
\end{flushleft}

\vspace{0pt plus2fill}

\begin{center}
Москва~--- 2026
\end{center}

\end{titlepage}
% Нумерация должна начинаться со второй страницы
\setcounter{page}{2}

% Оглавление
\newpage
\tableofcontents

% Введение
\newpage{}
\chapter*{Введение}
\addcontentsline{toc}{chapter}{Введение}
\subsubsection{Актуальность темы.} TODO

\subsubsection{Цели работы.} TODO

\subsubsection{Основные положения, выносимые на защиту.} TODO

\subsubsection{Методы исследования.} TODO

\subsubsection{Научная новизна.} TODO

\subsubsection{Теоретическая значимость.} TODO

\subsubsection{Практическая значимость.} TODO

\subsubsection{Степень достоверности и апробация работы.} TODO

\subsubsection{Публикации по теме диссертации.} TODO

\subsubsection{Личный вклад.} Все приведенные результаты, кроме отдельно оговоренных случаев, получены диссертантом лично при научном руководстве к.ф.-м.н. А.\,В.~Грабового.

\subsubsection{Структура и объем работы.} TODO

\subsubsection{Краткое содержание работы по главам.} TODO

% Первая глава
\clearpage
\chapter{Множество оптимальных параметров модели и проблема достаточности выборки}\label{chapter-1}
\input{chapter-1}

% Вторая глава
\clearpage
\chapter{Распределение оптимальных параметров в линейных моделях машинного обучения}\label{chapter-2}
\input{chapter-2}

% Третья глава
\clearpage
\chapter{Оптимизационная поверхность нейросетевых моделей и матрица Гессе как инструмент ее анализа}\label{chapter-3}
TODO

\section{Переход от линейных к нелинейным моделям}

TODO

\subsection{Ограничения подходов для линейных моделей}

TODO

\subsection{Специфика нейросетевых моделей: невыпуклость, вырожденность, симметрии}

TODO

\subsection{Необходимость новых подходов}

TODO

\section{Геометрия оптимизационной поверхности в нейронных сетях}

TODO

\subsection{Структура множества оптимальных параметров в глубоких сетях}

TODO

\subsection{Плоские минимумы и их связь с обобщающей способностью}

TODO

\subsection{Вырожденность и симметрии в пространстве параметров}

TODO

\subsection{Эмпирические наблюдения о сходимости поверхности функции потерь}

TODO

\section{Матрица Гессе как инструмент анализа оптимизационной поверхности}

TODO

\subsection{Гессиан функции потерь и его информативность}

TODO

\subsection{Спектральные свойства гессиана в нейронных сетях}

TODO

\subsection{Разложение Гаусса-Ньютона и его применимость}

TODO


% Четвертая глава
\clearpage
\chapter{Сходимость поверхности функции потерь при увеличении числа обучающих данных}\label{chapter-4}
\input{chapter-4}

% Пятая глава
\clearpage
\chapter{Методы оценки достаточного размера выборки}\label{chapter-5}
TODO

\section{Вычислительные методы для линейных моделей}

TODO

\subsection{Бутстрэппинг правдоподобия для D- и M-достаточности}

TODO

\subsection{Вычисление KL-расхождения для нормальных распределений}

TODO

\subsection{Вычисление s-score для сравнения распределений}

TODO

\subsection{Монте-Карло оценка апостериорных распределений}

TODO

\subsection{Практические алгоритмы реализации критериев}

TODO

\subsection{Относительные критерии достаточности}

TODO

\section{Методы для нейросетевых моделей}

TODO

\subsection{Оценка гессиана функции потерь: приближенные методы}

TODO

\subsection{Разложение Гаусса-Ньютона и его применение}

TODO

\subsection{Методы сэмплирования в пространстве параметров}

TODO

\subsection{Проекция на подпространство главных собственных векторов гессиана}

TODO

\subsection{Монте-Карло оценка распределенной сходимости}

TODO

\subsection{Эффективные алгоритмы для больших моделей}

TODO

\section{Критерии остановки сбора данных}

TODO

\subsection{Пороговые критерии на основе сходимости поверхности}

TODO

\subsection{Статистические тесты стабилизации}

TODO

\subsection{Адаптивные методы определения достаточности}

TODO

\subsection{Геометрический критерий достаточности выборки}

TODO

\section{Вычислительная сложность и оптимизация}

TODO

\subsection{Оценка вычислительной стоимости методов}

TODO

\subsection{Приближенные методы для ускорения вычислений}

TODO

\subsection{Проблема обращения ковариационных матриц}

TODO

\subsection{Масштабируемость методов для больших моделей}

TODO

\subsection{Практические рекомендации по применению}

TODO


% Шестая глава
\clearpage
\chapter{Вычислительные эксперименты и применения}\label{chapter-6}
TODO

\section{Экспериментальная валидация для линейных моделей}

TODO

\subsection{Синтетические данные: линейная и логистическая регрессия}

TODO

\subsection{Реальные датасеты: Liver Disorders, Boston Housing, Diabetes и др.}

TODO

\subsection{Сравнение различных критериев достаточности (D, M, KL, S)}

TODO

\subsection{Зависимость достаточного размера выборки от порога}

TODO

\subsection{Сравнение с существующими методами (Lagrange Multipliers, Wald Test и др.)}

TODO

\section{Эксперименты с полносвязными сетями}

TODO

\subsection{Задачи классификации: MNIST, FashionMNIST, CIFAR-10, CIFAR-100}

TODO

\subsection{Влияние архитектуры на сходимость поверхности: глубина и ширина}

TODO

\subsection{Связь между сходимостью поверхности и качеством модели (Accuracy)}

TODO

\subsection{Визуализация ландшафта функции потерь}

TODO

\subsection{Влияние batch normalization и dropout на оптимизационную поверхность}

TODO

\section{Эксперименты со сверточными сетями}

TODO

\subsection{Классификация изображений: MNIST, CIFAR-10}

TODO

\subsection{Анализ влияния глубины, ширины (числа каналов) и размера ядра}

TODO

\subsection{Влияние pooling-операций (MaxPool, AvgPool) и их позиции}

TODO

\subsection{Сравнение с полносвязными сетями}

TODO

\subsection{Визуализация ландшафта для сверточных архитектур}

TODO

\section{Распределенный подход: эксперименты с Monte Carlo}

TODO

\subsection{Влияние параметров распределения сэмплирования}

TODO

\subsection{Выбор размерности подпространства для анализа}

TODO

\subsection{Визуализация интегрируемой функции при различных размерах выборки}

TODO

\subsection{Сравнение точечного и распределенного подходов}

TODO

\subsection{Влияние архитектурных выборов на распределенную сходимость}

TODO

\section{Эксперименты с трансформерами}

TODO

\subsection{Анализ гессиана полного трансформера}

TODO

\subsection{Вклад различных компонентов (LayerNorm, Self-Attention, FFN)}

TODO

\subsection{Спектральные свойства гессиана трансформеров}

TODO

\subsection{Связь между архитектурой и свойствами оптимизационной поверхности}

TODO

\section{Практические применения}

TODO

\subsection{Определение необходимого объема данных для проекта}

TODO

\subsection{Оптимизация процесса сбора данных}

TODO

\subsection{Критерии остановки обучения}

TODO

\subsection{Геометрическая интерпретация достаточности выборки}

TODO


% Выводы
\clearpage
\chapter*{Заключение}
\addcontentsline{toc}{chapter}{Заключение}
\input{conclusion}

% \clearpage
% \chapter*{Общие свойства и определения}
% \addcontentsline{toc}{chapter}{Общие свойства и определения}
% \input{external}

% \clearpage
% \chapter*{Дополнительные Леммы и утверждения}
% \addcontentsline{toc}{chapter}{Дополнительные Леммы и утверждения}
% \input{extralemmas}

\clearpage 
\addcontentsline{toc}{chapter}{Список иллюстраций}
\listoffigures

\clearpage
\addcontentsline{toc}{chapter}{Список таблиц}
\listoftables

\clearpage
\addcontentsline{toc}{chapter}{Список литературы}
\renewcommand{\bibname}{Список литературы}

\nocite{*}
\bibliography{references}

\end{document}