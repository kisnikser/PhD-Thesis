TODO

\section{Экспериментальная валидация для линейных моделей}

TODO

\subsection{Синтетические данные: линейная и логистическая регрессия}

TODO

\subsection{Реальные датасеты: Liver Disorders, Boston Housing, Diabetes и др.}

TODO

\subsection{Сравнение различных критериев достаточности (D, M, KL, S)}

TODO

\subsection{Зависимость достаточного размера выборки от порога}

TODO

\subsection{Сравнение с существующими методами (Lagrange Multipliers, Wald Test и др.)}

TODO

\section{Эксперименты с полносвязными сетями}

TODO

\subsection{Задачи классификации: MNIST, FashionMNIST, CIFAR-10, CIFAR-100}

TODO

\subsection{Влияние архитектуры на сходимость поверхности: глубина и ширина}

TODO

\subsection{Связь между сходимостью поверхности и качеством модели (Accuracy)}

TODO

\subsection{Визуализация ландшафта функции потерь}

TODO

\subsection{Влияние batch normalization и dropout на оптимизационную поверхность}

TODO

\section{Эксперименты со сверточными сетями}

TODO

\subsection{Классификация изображений: MNIST, CIFAR-10}

TODO

\subsection{Анализ влияния глубины, ширины (числа каналов) и размера ядра}

TODO

\subsection{Влияние pooling-операций (MaxPool, AvgPool) и их позиции}

TODO

\subsection{Сравнение с полносвязными сетями}

TODO

\subsection{Визуализация ландшафта для сверточных архитектур}

TODO

\section{Распределенный подход: эксперименты с Monte Carlo}

TODO

\subsection{Влияние параметров распределения сэмплирования}

TODO

\subsection{Выбор размерности подпространства для анализа}

TODO

\subsection{Визуализация интегрируемой функции при различных размерах выборки}

TODO

\subsection{Сравнение точечного и распределенного подходов}

TODO

\subsection{Влияние архитектурных выборов на распределенную сходимость}

TODO

\section{Эксперименты с трансформерами}

TODO

\subsection{Анализ гессиана полного трансформера}

TODO

\subsection{Вклад различных компонентов (LayerNorm, Self-Attention, FFN)}

TODO

\subsection{Спектральные свойства гессиана трансформеров}

TODO

\subsection{Связь между архитектурой и свойствами оптимизационной поверхности}

TODO

\section{Практические применения}

TODO

\subsection{Определение необходимого объема данных для проекта}

TODO

\subsection{Оптимизация процесса сбора данных}

TODO

\subsection{Критерии остановки обучения}

TODO

\subsection{Геометрическая интерпретация достаточности выборки}

TODO
