\chapter{TODO}

TODO

\section{Формализация сходимости поверхности функции потерь}

TODO

\subsection{Точечный подход: определение сходимости в одной точке пространства параметров}

TODO

\subsection{Второй порядок аппроксимации и разложение Тейлора}

TODO

\subsection{Связь изменения функции потерь с гессианом}

TODO

\section{Полносвязные нейронные сети (MLP)}

TODO

\subsection{Разложение гессиана для полносвязных сетей}

TODO

\subsection{Оценка спектральной нормы гессиана через архитектуру сети}

TODO

\subsection{Теорема о скорости сходимости: зависимость от глубины и ширины}

TODO

\subsection{Связь с архитектурой сети: глубина, ширина, активации}

TODO

\section{Сверточные нейронные сети (CNN)}

TODO

\subsection{Матричное представление сверточных сетей}

TODO

\subsection{Специфика CNN: параметризация через матрицы сверток и Toeplitz-представление}

TODO

\subsection{Анализ гессиана в контексте сверточных операций}

TODO

\subsection{Теоретические оценки для 1D и 2D сверток}

TODO

\subsection{Влияние pooling-операций на оптимизационную поверхность}

TODO

\subsection{Полносвязные головы в сверточных сетях}

TODO

\section{Трансформеры и модели внимания}

TODO

\subsection{Архитектура трансформеров: LayerNorm, Self-Attention, Feed-Forward}

TODO

\subsection{Полный гессиан трансформера: учет взаимодействий компонентов}

TODO

\subsection{Анализ гессиана для моделей с механизмом внимания}

TODO

\subsection{Спектральные свойства и гетерогенность оптимизационной поверхности}

TODO

\subsection{Связь между величиной сигнала и остротой ландшафта}

TODO

\section{Распределенный подход к анализу сходимости}

TODO

\subsection{От точечного к распределенному анализу: обобщение подхода}

TODO

\subsection{Критерий сходимости по распределению: квадратичная разность в ожидании}

TODO

\subsection{Гауссовские окрестности локальных минимумов}

TODO

\subsection{Монте-Карло оценка сходимости оптимизационной поверхности}

TODO

\subsection{Теорема о скорости сходимости распределенного подхода}

TODO

\subsection{Сравнение точечного и распределенного подходов}

TODO

\section{Единый подход к различным архитектурам}

TODO

\subsection{Общие принципы анализа оптимизационной поверхности}

TODO

\subsection{Сравнение результатов для разных архитектур}

TODO

\subsection{Влияние архитектурных выборов: нормализация, dropout, глубина}

TODO

\subsection{Практические рекомендации по выбору критериев}

TODO
