\documentclass[12pt,a4paper]{article}

\input{preamble.tex}

\newsavebox{\coloredquotationbox}
\newenvironment{coloredquotation}
 {%
  \begin{trivlist}
  \begin{lrbox}{\coloredquotationbox}
  \begin{minipage}{\dimexpr\linewidth-2\fboxsep}
 }
 {%
  \end{minipage}
  \end{lrbox}
  \item\relax
  \parbox{\linewidth}{
    \begingroup
    \color[RGB]{224,215,188}%
    \hrule
    \color[RGB]{249,245,233}%
    \hrule
    \color[RGB]{224,215,188}%
    \hrule
    \endgroup
    \colorbox[RGB]{249,245,233}{\usebox{\coloredquotationbox}}\par\nointerlineskip
    \begingroup
    \color[RGB]{224,215,188}%
    \hrule
    \color[RGB]{249,245,233}%
    \hrule
    \color[RGB]{224,215,188}%
    \hrule
    \endgroup
  }
  \end{trivlist}
 }

\begin{document}
\pagestyle{empty}

\begin{center}
    \textsc{\large \textbf{Forecasting fMRI Images From Video Sequences:\\ Linear Model Analysis}}\\
    \vspace{0.5cm}
    \textsc{\large Point-by-point response}
\end{center}

\vspace{0.5cm}

We would like to express our sincere gratitude to all the reviewers for their valuable feedback and constructive criticism. Your insights and suggestions have been instrumental in helping us improve the quality and clarity of our manuscript. We appreciate the time and effort you have invested in reviewing our work, and we are committed to addressing each of your comments to enhance the overall presentation and substance of our research.

In the following response, we inform the reviewers that we have added the link at the our GitHub repository. If you think it the right thing to do, and it will not violate the confidentiality of checking our work, we leave it here: 

\href{https://github.com/DorinDaniil/Forecasting-fMRI-Images}{https://github.com/DorinDaniil/Forecasting-fMRI-Images}.

%%%%%%%%%%%%%%%%%%%%

\newpage
\subsection*{Reviewer 2}

Here we provide a point-by-point response for all comments received from the 2-nd reviewer.

\begin{coloredquotation}
In the method description section, the algorithm was not well integrated with the research content of this article. This section is more like simply introducing the algorithm itself.
\end{coloredquotation}

We appreciate the reviewer's feedback. We have revised the method description section to better integrate the algorithm with the research content. Specifically, we have explained how the linear autoregressive model and the use of ResNet152 for image embeddings are directly relevant to our goal of predicting fMRI images from video sequences. We have also provided more context on why these choices were made and how they contribute to the overall research objectives.

In addition, we have noticed in the text that from the point of view of studying the correlation between the video sequence and fMRI images, we need to use a simple model, preferably even a linear one.

\begin{coloredquotation}
In the experimental results section, although the author has analyzed the experimental results in multiple aspects, the experimental data is insufficient to support the author's viewpoint. And there was no comparison with other algorithms.
\end{coloredquotation}

We appreciate the reviewer's concern. However, it is important to note that the primary goal of our article is to study the dependence and correlation between the two multi-dimensional time series under investigation~--- specifically, the relationship between video sequences and fMRI images. Our focus is on understanding this fundamental correlation rather than comparing our method with other algorithms.

Regarding the amount of data, it is a well-known challenge in the field of mapping visual stimuli to fMRI images that there is a lack of high-quality datasets with detailed descriptions. The dataset we used, as described in the paper, is one of the most comprehensive available, containing the results of examinations of 63 subjects, with fMRI readings known for 30 of them. This dataset includes detailed annotations and is considered state-of-the-art for this type of research. Moreover, this dataset is of high interest for many researchers, as it was presented only 2 years ago.

While we understand the importance of comparisons with other algorithms, our current focus is on establishing the basic correlation and understanding the underlying dependencies. Future work will aim to expand the dataset and include comparisons with other methods to provide a more comprehensive evaluation.

In addition, we added a link to GitHub with the sources, where the reader can verify the correctness of our results.

\begin{coloredquotation}
The writing style of this article is not like a paper, such as the use of "Let's". There are also some formatting and textual issues, such as the numbering of formulas.
\end{coloredquotation}

We appreciate the reviewer's attention to detail. We have revised the writing style to be more formal and appropriate for a research paper. We have removed informal language such as 'Let's' and ensured that the text adheres to academic standards. Additionally, we have corrected any formatting and textual issues, including the numbering of formulas, to ensure consistency and clarity throughout the paper.

%%%%%%%%%%%%%%%%%%%%

\newpage
\subsection*{Reviewer 5}

Here we provide a point-by-point response for all comments received from the 5-th reviewer.

\begin{coloredquotation}
There is a need for a clearer and more explicit statement of the research purpose, background and significance of the article in order to make it easier for the reader to understand.
\end{coloredquotation}

We appreciate the reviewer's suggestion. We have revised the introduction section to provide a clearer and more explicit statement of the research purpose, background, and significance. We have highlighted the importance of understanding the relationship between fMRI images and video sequences and explained how our research contributes to this field. Specifically, our focus is on understanding this fundamental correlation rather than comparing our method with other algorithms.

This should make it easier for readers to understand the context and significance of our work. Moreover, we have added a Related Work section, in which we described the present research on the topic under consideration.

\begin{coloredquotation}
The section method needs more details, for example, what is the purpose of choosing ResNet152 and why a linear model is used.
\end{coloredquotation}

We appreciate the reviewer's feedback. We have added more details to the method section to explain the rationale behind our choices. Specifically, we have explained that ResNet152 was chosen for its high accuracy, efficiency and ability to capture complex image features, which are crucial for our task of predicting fMRI images from video sequences. Actually, there is no difference what image encoder to choose: ResNet152 or ViT, but for the best possible fMRI images reconstructions, it may have a particular sense. We will expand our research on this point in the future work.

We have also explained why a linear model was used, highlighting its simplicity, interpretability, and effectiveness in capturing the correlation between video frames and fMRI images.

\begin{coloredquotation}
More evaluation metrics are also needed, if possible.
\end{coloredquotation}

We appreciate the reviewer's suggestion. However, we believe that our experiments are precise and comprehensive enough to understand the dependency and correlation between visual stimuli and fMRI images. Given the primary goal of our study, which is to establish the fundamental correlation between the two time series, we believe that the current evaluation metrics are sufficient.

We understand the importance of additional metrics in providing a more thorough evaluation, and we will consider including them in future work as we expand our research and datasets.

\begin{coloredquotation}
All in all, the article also needs more details about your work.
\end{coloredquotation}

We appreciate the reviewer's feedback. We have added more details throughout the paper to provide a more comprehensive description of our work. This includes a detailed explanation in the introduction section, additional explanations in the method section, and a more thorough discussion of our findings and their implications. These additions should help to provide a clearer and more complete picture of our research.

\end{document}
